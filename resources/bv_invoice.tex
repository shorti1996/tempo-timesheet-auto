\listfiles
\documentclass{article}
\usepackage[T1]{fontenc}
\usepackage{lmodern}
\usepackage[utf8]{inputenc}
\usepackage{natbib}
\usepackage{fancyhdr}
\usepackage{graphicx}
\usepackage{titlesec}
\usepackage{adjustbox}
\usepackage[table]{xcolor}
% \usepackage{tabu}
\usepackage{tabularx}
\usepackage{makecell}
\usepackage{booktabs}
\usepackage[a4paper, total={6in, 8in}]{geometry}
\usepackage{array,multirow}
\usepackage{multicol}
\geometry{
    a4paper,
    total={170mm,257mm},
    top=40mm,
    bottom=40mm,
    left=20mm,
    right=20mm,
    headheight=50pt,
    headsep=40pt,
}
\usepackage{array,etoolbox}
\preto\tabular{\setcounter{magicrownumbers}{0}}
\newcounter{magicrownumbers}
\newcommand\rownumber{\stepcounter{magicrownumbers}\arabic{magicrownumbers}}

\pagestyle{fancy}
\pagestyle{fancy}
\fancyhf{} % sets both header and footer to nothing
\renewcommand{\headrulewidth}{0pt}


\titleformat
{\section} % command
[block] % shape
{\bfseries\Large\normalfont} % format
{\thesection } % label
{0.5ex} % sep
{
    \centering
    \uppercase
} % before-code

\titlespacing{\section}{12pc}{1.5ex plus .1ex minus .2ex}{1pc}

\newcommand{\bsgpipe}{
    \textbf{\textcolor{red}{|}}
}

\newcommand{\rowspace}{
    \begin{minipage}[c][2em][t]{0pt}\end{minipage}
}

\newcommand{\VAR}[1]{
    \textbf{\textcolor{red}{#1}}
}

\begin{document}

    \newpage

    \begin{flushright}
        \textbf{Invoice No.} / faktura nr: \\
        \VAR{invoiceNumber} \\
        \vspace{0.5em}
        \textbf{Invoice date} / data wystawienia: \\
        \VAR{invoiceDate} \\
        \vspace{0.5em}
        \textbf{Transaction date} / data wydania towaru lub wykonania usługi: \\
        \VAR{transactionDate} \\
    \end{flushright}

    \vspace{1em}

    % \begin{adjustbox}{width=1\textwidth,center}
    \begin{center}
        \begin{tabular}{l l}
        \textbf{Supplier} / wystawca: & \textbf{Client} / nabywca: \\
        \\
        \VAR{supplierDataLine1} & \VAR{clientDataLine1} \\
        \VAR{supplierDataLine2} & \VAR{clientDataLine2} \\
        \VAR{supplierDataLine3} & \VAR{clientDataLine3} \\
        \\
        \textbf{VAT code} / NIP: \VAR{supplierVatCode} & \textbf{VAT code} / NIP: \VAR{clientVatCode}
        \end{tabular}
    \end{center}
    % \end{adjustbox}

    % \begin{multicols}{2}
    % \noindent Supplier / wystawca: \\
    % Wojciech Liebert \\
    % Dobra 179a/2 \\
    % 59-700 Bolesławiec, Polska \\
    % VAT code / NIP: PL6121826127 \\
    % \vfill\null
    % \columnbreak

    % \noindent Client / nabywca: \\
    % blue veery GmbH \\
    % Gotzinger Str. 46 8137 \\
    % 81371 München, Niemcy \\
    % VAT code / NIP: DE301666398 \\
    % \end{multicols}

    \vspace{2em}

    % \renewcommand\theadalign{bc}
    % \renewcommand\theadfont{\bfseries}
    % \renewcommand\theadgape{\Gape[4pt]}
    \renewcommand\cellgape{\Gape[4pt]}

    \begin{adjustbox}{width=1\textwidth,center}
        \begin{tabular}{|c|c|c|c|c|c|c|c|c|c|}
          \hline
          \thead{\textbf{No.} \\ L.p.} & \thead{\textbf{Description} \\ Opis} & \thead{\textbf{Net price} \\ Cena netto} & \thead{\textbf{Q-ty} \\ Ilość} & \thead{\textbf{Unit} \\ Jedn.} & \thead{\textbf{Net amount} \\ Wartość netto} & \thead{\textbf{VAT\%}} & \thead{\textbf{VAT amount} \\ Wartość netto} & \thead{\textbf{Gross amount} \\ Wartość brutto} \\
          \hline
          \makecell{1} & \makecell[l]{\VAR{description}}  & \makecell{\VAR{netPrice}} & \makecell{1} & \makecell{usł.} & \makecell{\VAR{netAmount}} & \makecell{NP} & \makecell{\VAR{vatAmount}} & \makecell{\VAR{grossAmount}} \\
          \hline
          \multicolumn{3}{l}{} & \makecell{} & \makecell{\textbf{Total:} \\ Razem:} & \makecell{\VAR{netAmount}}  & \makecell{---} & \makecell{\VAR{vatAmount}} & \makecell{\VAR{grossAmount}} \\
          \cline{5-9}
          \multicolumn{3}{l}{} & \makecell{} & \makecell{\textbf{Subtotal:} \\ W tym: (PLN)} & \makecell{\VAR{netAmount}}  & \makecell{NP} & \makecell{\VAR{vatAmount}} & \makecell{\VAR{grossAmount}} \\
          \cline{5-9}
        \end{tabular}
    \end{adjustbox}

    \renewcommand\cellgape{\Gape[1pt]}
    \begin{flushright}
        \vspace{3em}
        % \begin{adjustbox}{width=1\textwidth,center}
            \begin{tabular}{|l|r|}
              \hline
              \makecell[l]{\textbf{Total amount due} / do zapłaty:} & \makecell[r]{\VAR{grossAmount}} \\
              \hline
              \makecell[l]{\textbf{In words:}} & \makecell[r]{\VAR{grossAmountWordsEn} \VAR{grossAmountHundredths}/100 PLN} \\
              \hline
              \makecell[l]{Słownie:} & \makecell[r]{\VAR{grossAmountWordsPl} \VAR{grossAmountHundredths}/100 PLN} \\
              \hline
              \makecell[l]{\textbf{Payment by} / sposób zapłaty:} & \makecell[r]{transfer/ przelew} \\
              \hline
              \makecell[l]{\textbf{Payment due by} / termin:} & \makecell[r]{\VAR{paymentDue}} \\
              \hline
              \makecell[l]{\textbf{Our bank account} / rachunek:} & \makecell[r]{\VAR{bankAccountNumber}} \\
              \hline
              \makecell[l]{\textbf{BIC / SWIFT}} & \makecell[r]{\VAR{bicSwiftCode}} \\
              \hline
            \end{tabular}
        % \end{adjustbox}
    \end{flushright}

    \vspace{1em}
    \noindent \textbf{Reverse charge} / odwrotne obciążenie \\
    \textbf{Comments} / uwagi:

\end{document}